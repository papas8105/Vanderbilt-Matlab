\documentclass[10pt]{article}
\usepackage{amsmath}
\usepackage{xltxtra}
\usepackage{unicode-math}
\setmathfont{Asana-Math.otf}
\usepackage{pstricks-add}
\usepackage{geometry}
\geometry{
 a4paper,
 total={210mm,297mm},
 left=10mm,
 right=10mm,
 top=15mm,
 bottom=15mm,
 }
 \usepackage{xcolor,colortbl}\usepackage{multirow}
\setmainfont[Mapping=tex-text]{Garamond Premier Pro}
\usepackage{graphicx}\usepackage{sidecap}\usepackage{bbm}
\title{\textbf{Brief Explanation Of The Code In ''2016/hw8/maxproduct.m''}}
\author{George Papadopoulos}\date{pgeorgios8@gmail.com}
\newcommand{\dsp}{\displaystyle}
\setlength\parindent{0pt}
\usepackage{pstricks-add}
\usepackage[framed,numbered,autolinebreaks,useliterate]{mcode}
\usepackage{hyperref}
\hypersetup{colorlinks=true,urlcolor=cyan}
\begin{document}
 \maketitle
 \begin{abstract}
In response to an email asking clarifications for  the code in 
''2016/hw8/maxproduct.m'',  
in this small informal article we explore the  solution of the fifth problem and 
the algorithmic process that leads to its solution more thoroughly. It is the author's 
view, meaning ''my opinion'' that this is the most difficult problem  of the eighth 
module and the most time consuming to solve throughout the course.\par
This article is divided into  two sections, describing the two 
steps of this solution and the abstract thinking behind the first step that led to the 
second.\par
The problem we are asked to solve is to write a program with two inputs, a 
matrix \textit{A} and a \textbf{positive} integer \textit{n} (radius) that gives as 
output the indexes of the maximum product of n consecutive elements of the matrix in all 
directions, meaning row, column, diagonal and cross diagonal directions, firstly in 
ascending  row wise order and secondly 
if all the elements belong to the 
same row in ascending column wise 
order. If no such product exists, the 
output must return the empty array 
and in the case of multiple maximum 
products the output must return the 
first one found.[\href{https://projecteuler.net/problem=11}{Problem 11 - Euler Project} ]
 \end{abstract}
In the two following sections for simplicity reasons \footnote{$\rm\TeX$ typesetting 
might be difficult some times believe me.} we will use the $2\times2$ matrix \[
 A=\left[\begin{matrix}2 & 3 \\ 4 & 1\end{matrix}\right]\] for which is obvious that the 
output we should expect is $[1,2]$ and $[2,1]$, since $A(1,2)\cdot A(2,1) = 12$ which is 
the maximum of all the possible products of two consecutive elements in this matrix.
\section{String Assignment.} The problem we should first consider is that some elements 
miss some directions that lead to an error \footnote{Index  exceeds matrix dimensions.}, 
since {\color{red}\textbf{matlab}} expects only valid operations to be conducted and 
expects from the end user to respect that principle, for example we cannot get the third 
element of a two dimensional array. The first information we are given  is that we should 
search in eight directions diagonal, cross diagonal, horizontal and vertical, the second 
information is that we should compute in the valid directions (the ones that exist) the 
product of n  consecutive elements and store it ''somewhere'' along with the indexes, 
then when all the elements of the matrix are scanned, go back to find where the maximum 
product was found in the matrix.\par
Suppose we are ''somewhere'' on a dangerous terrain and the only tool we have in our 
hands is a compass with eight directions starting from north west in a clockwise 
direction each denoted by two letters  NW, NO, NE, EA, SE, SO, SW, WE 
\footnote{north-west, north, north-east, east, south-east, south, south-west, west}  and 
a piece of paper with some letters and numbers next to them like the following 
\begin{center}\begin{tabular}{c|r}Direction & Steps \\ NE & 3\\NO&2\\WE & 
2\end{tabular}\end{center}that give us the safe path we can walk on in order  to avoid 
the pits and find the exit. We can imagine that at each point we will have to  stop, we 
are standing on an element of a matrix around which the ''x'' denotes a point of the path 
and 
the ''o'' a pit we should avoid,\vspace{15pt}
\begin{center}
\begin{tabular}{|c|c|c|c|c||c|}
\hline
o & o & x & x & exit &compass\\\hline
o & o & x & o & 
o&\multirow{4}{*}{\scalebox{0.5}{\psset{xunit=0.6cm,yunit=0.6cm,algebraic=true,
dimen=middle , dotstyle=o , dotsize=5pt 
0,linewidth=1.6pt,arrowsize=3pt 2,arrowinset=0.25}
\begin{pspicture*}(-3,-3)(3,3)
\psline[linewidth=2.pt]{->}(0.,-2.)(0.,2.)
\psline[linewidth=2.pt]{->}(-2.,0.)(2.,0.)
\psline[linewidth=2.pt]{->}(-1.,-1.)(1.,1.)
\psline[linewidth=2.pt]{->}(1.,-1.)(-1.,1.)
\psline[linewidth=2.pt]{->}(0.,2.)(0.,-2.)
\psline[linewidth=2.pt]{->}(2.,0.)(-2.,0.)
\psline[linewidth=2.pt]{->}(-1.,1.)(1.,-1.)
\psline[linewidth=2.pt]{->}(1.,1.)(-1.,-1.)
\rput[tl](-0.4,2.45){NO}
\rput[tl](1,1.4){NE}
\rput[tl](-1.6,1.4){NW}
\rput[tl](2.1,0.12){EA}
\rput[tl](1.05,-1.124){SE}
\rput[tl](-0.4,-2.15){SO}
\rput[tl](-1.6,-1.124){SW}
\rput[tl](-3,0.12){WE}
\end{pspicture*}}}\\\cline{1-5}
o & o & x & o &o&\\\cline{1-5}
o & x & o & o & o&\\\cline{1-5}
x & o & o & o &o&\\\hline
\end{tabular}
\end{center}% 2 3 4 1 Christine 
\vspace{15pt}
so what if we followed the same logic in our problem? Suppose we are standing at position 
$(2,1)$ of matrix A, with the same logic described above we can imagine an image like the 
one below
\newcommand{\mc}[3]{\multicolumn{#1}{#2}{#3}}
\definecolor{tcA}{rgb}{1,0,0}
\begin{center}
\begin{tabular}{cccclll}
\textcolor{tcA}{o} & \textcolor{tcA}{o} & \textcolor{tcA}{o} & \textcolor{tcA}{o} &  & 
& \\\cline{2-3}
\mc{1}{c|}{\textcolor{tcA}{o}} & \mc{1}{c|}{2} & \mc{1}{c|}{3} & \textcolor{tcA}{o} &  & 
NO & NE\\\cline{2-3}
\mc{1}{c|}{\textcolor{tcA}{o}} & \mc{1}{c|}{4} & \mc{1}{c|}{1} & \textcolor{tcA}{o} & 
$\longrightarrow$ & we are here valid moves for A(2,1) & EA\\\cline{2-3}
\textcolor{tcA}{o} & \textcolor{tcA}{o} & \textcolor{tcA}{o} & \textcolor{tcA}{o} &  & 
& 
\end{tabular}
\end{center}
so we can write the \textit{valid} directions for $A(2,1)$ a a sting sequence like 
'NONEEA' which is translated into north, north-east, east. What if we used a \text{fixed} 
string with all the directions starting from NW in the clockwise direction? So let 
$\text{directions}\;=\;''\hspace{-2.6pt}NWNONEEASESOSWWE''$, which is a 16-bit string. 
For every element of a matrix we can check each direction by getting two bits at a time 
of the string ''directions'' and replace them if there is no path of length n to the 
specified direction with an $''xx''$ 2-bit string. So if we found a way understood by the 
computer to do the checking for $A(1,1)$ for $n=2$ we would get the string 
$''xxxxxxEASESOxxxx''\;\dagger$, meaning that the valid paths are east, south-east and 
south. An easy way to do it with {\color{red}matlab} would be to create a cell  of 
the same size of matrix $A$ and assign to its empty positions the outputs of the form 
$\dagger$. For our case a schema showing the above description would be
\vspace{-10pt}
\begin{center}
\begin{pspicture}(0,-2.42)(9.151563,2.42)
\definecolor{color235}{rgb}{0.9764705882352941,0.03529411764705882,0.023529411764705882}
\definecolor{color236}{rgb}{1.0,0.0,0.03529411764705882}
\psline[linewidth=0.04](0.22,0.61)(0.0,0.61)(0.0,-0.77)(0.2,-0.77)(0.08,-0.77)
\usefont{T1}{ptm}{m}{n}
\rput(1.1676563,0.355){3}
\usefont{T1}{ptm}{m}{n}
\rput(0.5409375,-0.405){4}
\usefont{T1}{ptm}{m}{n}
\rput(0.53859377,0.355){2}
\usefont{T1}{ptm}{m}{n}
\rput(1.146875,-0.405){1}
\psline[linewidth=0.04](1.4,0.61)(1.62,0.61)(1.62,-0.73)(1.42,-0.73)
\psbezier[linewidth=0.02,linecolor=red,arrowsize=0.05291667cm 
2.0,arrowlength=1.4,arrowinset=0.4]{->>}(0.62,0.51)(2.3518436,1.87)(3.78,0.33)(3.92,0.13)
\psbezier[linewidth=0.02,linecolor=red,arrowsize=0.05291667cm 
2.0,arrowlength=1.4,arrowinset=0.4]{->>}(1.3,0.5570134)(1.3,0.3782243)(4.3,2.41)(7.78,
0.05)
\usefont{T1}{ptm}{m}{n}
\rput(3.9154687,-0.225){xxxxxxEASESOxxxx}
\usefont{T1}{ptm}{m}{n}
\rput(7.5134373,-0.225){xxxxxxxxxxSOSWWE}
\usefont{T1}{ptm}{m}{n}
\rput(3.9554687,-0.685){xxNONEEAxxxxxxxx}
\usefont{T1}{ptm}{m}{n}
\rput(7.478125,-0.665){NWNOxxxxxxxxxxWE}
\psbezier[linewidth=0.02,linecolor=color235,arrowsize=0.05291667cm 
2.0,arrowlength=1.4,arrowinset=0.4]{->>}(0.56,-0.59)(0.7168705,-0.65529794)(1.4983051,
-1.83)(4.02,-0.9659692)
\psbezier[linewidth=0.02,linecolor=color236,arrowsize=0.05291667cm 
2.0,arrowlength=1.4,arrowinset=0.4]{->>}(1.24,-0.53)(3.76,-2.41)(6.54,-1.47)(7.78,-0.93)
\end{pspicture} 
\end{center}
\vspace{-10pt}
and the code in {\color{red}matlab} which illustrates what is described here is the 
following.
\begin{lstlisting}
function v = maxproduct(A,n)
assert(isscalar(n) && isnumeric(n) && n >= 0 && ceil(n) == n,'error radius must be greater 
than zero!');
c = cell(size(A,1),size(A,2));
for ii = 1 : size(A,1)
    for jj = 1 : size(A,2)
        directions = 'NWNONEEASESOSWWE';
        try %NW
            A(ii-n+1:ii,jj-n+1:jj);
        catch
            directions(1:2) = 'xx';
        end
        try %NO
            A(ii-n+1:ii,jj);
        catch
            directions(3:4) = 'xx';
        end
        try %NE
            A(ii-n+1:ii,jj:jj+n-1);
        catch
            directions(5:6) = 'xx';
        end
        try %EA
            A(ii,jj:jj+n-1);
        catch
            directions(7:8) = 'xx';
        end
        try %SE
            A(ii:ii+n-1,jj:jj+n-1);
        catch
            directions(9:10) = 'xx';
        end
        try %SO
            A(ii:ii+n-1,jj);
        catch
            directions(11:12) = 'xx';
        end
        try %SW
            A(ii:ii+n-1,jj-n+1:jj);
        catch
            directions(13:14) = 'xx';
        end
        try %WE
            A(ii,jj-n+1:jj);
        catch
            directions(15:16) = 'xx';
        end
        c{ii,jj} = directions;
    end
end
v = c;
end
\end{lstlisting}
We can test this function for various matrices with various $n>0$ values and this 
concludes section 1 having solved the problem of the invalid directions for each matrix 
element.\\
Examples:
\begin{lstlisting}
>>maxproduct([2 3;4 1],2)

 ans =
    'xxxxxxEASESOxxxx'    'xxxxxxxxxxSOSWWE'
    'xxNONEEAxxxxxxxx'    'NWNOxxxxxxxxxxWE'
    
 >>class(ans)
 
 ans = 

    cell
    
>>maxproduct([1 2 3 ; 4 5 6;7 8 9],3)

 ans = 

    'xxxxxxEASESOxxxx'  'xxxxxxxxxxSOxxxx'    'xxxxxxxxxxSOSWWE'
    'xxxxxxEAxxxxxxxx'  'xxxxxxxxxxxxxxxx'    'xxxxxxxxxxxxxxWE'
    'xxNONEEAxxxxxxxx'  'xxNOxxxxxxxxxxxx'    'NWNOxxxxxxxxxxWE'
    
>>class(ans)

 ans =
  
    cell
\end{lstlisting}
\section{Maximum Product And Indexes.}
For the second part of the program, we should be aware that the output must sort the 
indexes in a row wise ascending order and secondly in a column wise ascending order. Here 
is a figure to  give us an impression of how we should sort the indexes, the 
arrows' tails show to us the first term's indexes in the product we compute (if there 
exists an element in that direction and the direction is not marked as ''xx'') and the 
arrow's head the element which is  scanned at each time (imagine a planet with its 
satellites).
\begin{center}
\newrgbcolor{ttqqqq}{0.2 0. 0.}
\psset{xunit=0.6cm,yunit=0.6cm,algebraic=true,dimen=middle,dotstyle=o,dotsize=5pt 
0,linewidth=1.6pt,arrowsize=3pt 2,arrowinset=0.25}
\begin{pspicture*}(-6.776363636363642,-7.40181818181816)(10.593636363636364,
6.578181818181798)
\rput[tl](0.5,-0.05){$a_{ii,jj}$}
\rput[tl](0.2136363636363604,1.4181818181818133){$a_{ii-1,jj}$}
\rput[tl](-0.05636363636363969,3.068181818181808){$a_{ii-n+1,jj}$}
\rput[tl](0.2436363636363604,-0.7418181818181802){$a_{ii+1,jj}$}
\rput[tl](0.03363636363636034,-2.6918181818181743){$a_{ii+n-1,jj}$}
\rput[tl](-5.096363636363641,0.2481818181818168){$a_{ii,jj-n+1}$}
\rput[tl](-1.9163636363636403,0.2781818181818167){$a_{ii,jj-1}$}
\rput[tl](2.043636363636361,0.3081818181818166){$a_{ii,jj+1}$}
\rput[tl](5.223636363636362,0.3081818181818166){$a_{ii,jj+n-1}$}
\rput[tl](-1.9763636363636403,1.4481818181818131){$a_{ii-1,jj-1}$}
\rput[tl](-5.066363636363641,3.098181818181808){$a_{ii-n+1,jj-n+1}$}
\rput[tl](-5.006363636363641,-2.6918181818181743){$a_{ii+n-1,jj-n+1}$}
\rput[tl](-1.9463636363636403,-0.7418181818181802){$a_{ii+1,jj-1}$}
\rput[tl](5.193636363636362,3.068181818181808){$a_{ii-n+1,jj+n-1}$}
\rput[tl](5.223636363636362,-2.5718181818181747){$a_{ii+n-1,jj+n-1}$}
\rput[tl](2.073636363636361,1.3881818181818133){$a_{ii-1,jj+1}$}
\rput[tl](2.103636363636361,-0.7118181818181802){$a_{ii+1,jj+1}$}
\psline[linewidth=0.8pt,linestyle=dashed,dash=2pt 2pt,linecolor=red]{->}(-6.,4.)(0.,0.)
\psline[linewidth=0.8pt,linestyle=dashed,dash=4pt 
4pt,linecolor=red]{->}(1.0236363636363606,4.028181818181805)(1.0836363636363606,
-0.021818181818182403)
\psline[linewidth=0.8pt,linestyle=dashed,dash=4pt 4pt,linecolor=red]{->}(8.,4.)(2.,0.)
\psline[linewidth=0.8pt,linestyle=dashed,dash=4pt 
4pt,linecolor=red]{->}(-6.026363636363642,-0.3818181818181813)(0.09363636363636019,
-0.4418181818181811)
\psline[linewidth=0.8pt,linestyle=dashed,dash=4pt 
4pt,linecolor=red]{<-}(-6.,-4.)(0.003636363636360329,-0.8918181818181798)
\psline[linewidth=0.8pt,linestyle=dashed,dash=4pt 
4pt,linecolor=red]{<-}(1.1436363636363607,-4.10181818181817)(1.0836363636363606,
-0.9818181818181797)
\psline[linewidth=0.8pt,linestyle=dashed,dash=4pt 
4pt,linecolor=red]{<-}(8.,-4.)(2.013636363636361,-0.6518181818181805)
\psline[linewidth=0.8pt,linestyle=dashed,dash=4pt 
4pt,linecolor=red]{<-}(7.953636363636363,-0.32181818181818145)(1.9536363636363614,
-0.3818181818181813)
\end{pspicture*}
\end{center}
The question we should next answer is how  we should compute the product in each 
direction and how we should store the indexes, for which we got a glimpse of how they 
should be sorted correctly. Let us examine each direction separately:\\
\textbf{{\color{red}NW.}}\\
In this case the product is the product of the elements of the diagonal of the 
north-western $n\times n$ matrix with its lowest right element being $a_{ii,jj}$ and this 
is as written 
in {\color{red}matlab} as $A(ii-n+1:ii,jj-n+1:jj)$. 
\begin{lstlisting}
product = prod(diag(A(ii-n+1:ii,jj-n+1:jj)))
indexes = [(ii-n+1:ii)',(jj-n+1:jj)']  
\end{lstlisting}
\textbf{{\color{red}NO.}}\\
In this case the product is the product of the elements of the array in the northern
direction $n\times 1$ with its last element being $a_{ii,jj}$ and this is written in 
{\color{red}matlab} as $A(ii-n+1:ii,jj)$. 
\begin{lstlisting}
product = prod(A(ii-n+1:ii,jj)))
indexes = [(ii-n+1:ii)',repmat(jj,n,1)]  
\end{lstlisting}
\textbf{{\color{red}NE.}}\\
In this case the product is the product of the elements of the reverse diagonal of the 
north-eastern $n\times n$ matrix with its lowest left element being $a_{ii,jj}$ and this 
is 
 written in {\color{red}matlab} as $A(ii-n+1:ii,jj:jj+n-1)$. In order to compute the 
product we will have to flip the matrix upside down and take the indexes from right to 
left in a column wise direction.
\begin{lstlisting}
product = prod(diag(flipud(A(ii-n+1:ii,jj:jj+n-1))))
indexes = [(ii-n+1:ii)',(jj+n-1:-1:jj)']  
\end{lstlisting}
\textbf{{\color{red}EA.}}\\
In this case the product is the product of the elements of the array of length $n$ in the 
eastern direction with its first left element 
being $a_{ii,jj}$ and this is as written in {\color{red}matlab} as $A(ii,jj:jj+n-1)$. 
\begin{lstlisting}
product = prod(A(ii,jj:jj+n-1))
indexes = [repmat(ii,n,1),(jj:jj+n-1)']  
\end{lstlisting}
\textbf{{\color{red}SE.}}\\
In this case the product is the product of the elements of the diagonal of the 
south-eastern $n\times n$ matrix with its upper left element being $a_{ii,jj}$ and this 
is as written in {\color{red}matlab} $A(ii:ii+n-1,jj:jj+n-1)$. 
\begin{lstlisting}
product = prod(diag(A(ii:ii+n-1:-1,jj:jj+n-1:-1)))
indexes = [(ii:ii+n-1)',(jj:jj+n-1)']  
\end{lstlisting}
\textbf{{\color{red}SO.}}\\
In this case the product is the product of the elements of the array in the southern
direction $n\times 1$ with its first element being $a_{ii,jj}$ and this is written in 
{\color{red}matlab} as $A(ii:ii+n-1,jj)$. 
\begin{lstlisting}
product = prod(A(ii:ii+n-1,jj)))
indexes = [(ii:ii+n-1)',repmat(jj,n,1)] 
\end{lstlisting}
\textbf{{\color{red}SW.}}\\
In this case the product is the product of the elements of the reverse diagonal of the 
south-western $n\times n$ matrix with its upper right element being $a_{ii,jj}$ and this 
is written in {\color{red}matlab} as $A(ii:ii+n-1,jj-n+1:jj)$. In order to compute the 
product we will have to flip the matrix upside down and take the indexes from right to 
left in a column wise direction.
\begin{lstlisting}
product = prod(diag(flipud(A(ii:ii+n-1,jj-n+1:jj))))
indexes = [(ii:ii+n-1)',(jj:-1:jj-n+1)']  
\end{lstlisting}
\textbf{{\color{red}WE.}}\\
In this case the product is the product of the elements of the array of length $n$ in the 
western direction with its last right element being $a_{ii,jj}$ and this is as written in 
{\color{red}matlab} as $A(ii,jj-n+1:jj)$. 
\begin{lstlisting}
product = prod(A(ii,jj-n+1:jj))
indexes = [repmat(ii,n,1),(jj-n+1:jj)']  
\end{lstlisting}
and now that we know what to do with the indexes and the products in each direction we 
are ready for the final step. We can assign an initial value $p=0$ to the product  and 
an empty initial array of indexes $v = []$, such that if no product greater 
than zero is found, then the output of the program shall return the empty array. We must 
iterate through all the elements of cell $c$ created in the first section, get the string 
of each position, split the string into 8 parts beginning from left to right by step two, 
see which match a valid direction meaning not equal to ''xx'', then compute the product 
and finally if that product is greater than $p$ set $v$ equal to the indexes of that 
combination of terms. In this case the ''switch-case'' function turns to be useful, this 
gives us the code presented the next page and our program is ready!\pagebreak
\begin{lstlisting}
function v = maxproduct(A,n)
assert(isscalar(n) && isnumeric(n) && n >= 0 && ceil(n) == n,'error radius must be greater 
than zero!');
c = cell(size(A,1),size(A,2));
for ii = 1 : size(A,1)
    for jj = 1 : size(A,2)
        directions = 'NWNONEEASESOSWWE';
        try %NW
            A(ii-n+1:ii,jj-n+1:jj);
        catch
            directions(1:2) = 'xx';
        end
        try %NO
            A(ii-n+1:ii,jj);
        catch
            directions(3:4) = 'xx';
        end
        try %NE
            A(ii-n+1:ii,jj:jj+n-1);
        catch
            directions(5:6) = 'xx';
        end
        try %EA
            A(ii,jj:jj+n-1);
        catch
            directions(7:8) = 'xx';
        end
        try %SE
            A(ii:ii+n-1,jj:jj+n-1);
        catch
            directions(9:10) = 'xx';
        end
        try %SO
            A(ii:ii+n-1,jj);
        catch
            directions(11:12) = 'xx';
        end
        try %SW
            A(ii:ii+n-1,jj-n+1:jj);
        catch
            directions(13:14) = 'xx';
        end
        try %WE
            A(ii,jj-n+1:jj);
        catch
            directions(15:16) = 'xx';
        end
        c{ii,jj} = directions;
    end
end
p = 0;v = [];
for ii = 1 : size(c,1)
    for jj = 1 : size(c,2)
        el = c{ii,jj};
        for kk = 1 : 8
            input = el(2*kk-1:2*kk);
            switch input
                case 'NW'
                    pr    = prod(diag(A(ii-n+1:ii,jj-n+1:jj)));
                    ind  = [(ii-n+1:ii)',(jj-n+1:jj)'];
                    if pr > p
                        v = ind;
                        p = pr;
                    end
                case 'NO'
                    pr    = prod(A(ii-n+1:ii,jj));
                    ind  = [(ii-n+1:ii)',repmat(jj,n,1)];
                    if pr > p;
                        v = ind;
                        p = pr;
                    end
                case 'NE'
                    pr    = prod(diag(flipud(A(ii-n+1:ii,jj:jj+n-1))));
                    ind  = [(ii-n+1:ii)',(jj+n-1:-1:jj)'];
                    if pr > p
                        v = ind;
                        p = pr;
                    end
                case 'EA'
                    pr    = prod(A(ii,jj:jj+n-1));
                    ind  = [repmat(ii,n,1),(jj:jj+n-1)'];
                    if pr > p
                        v = ind;
                        p = pr;
                    end
                case 'SE'
                    pr    = prod(diag(A(ii:ii+n-1,jj:jj+n-1)));
                    ind  = [(ii:ii+n-1)',(jj:jj+n-1)'];
                    if pr > p
                        v = ind;
                        p = pr;
                    end
                case 'SO'
                    pr    = prod(A(ii:ii+n-1,jj));
                    ind  = [(ii:ii+n-1)',repmat(jj,n,1)];
                    if pr > p
                        v = ind;
                        p = pr;
                    end
                case 'SW'
                    pr    = prod(diag(flipud(A(ii:ii+n-1,jj-n+1:jj))));
                    ind  = [(ii:ii+n-1)',(jj:-1:jj-n+1)'];
                    if pr > p
                        v = ind;
                        p = pr;
                    end
                case 'WE'
                    pr    = prod(A(ii,jj-n+1:jj));
                    ind  = [repmat(ii,n,1),(jj-n+1:jj)'];
                    if pr > p
                        v = ind;
                        p = pr;
                    end
            end
        end
    end
end
end
\end{lstlisting}
\end{document}